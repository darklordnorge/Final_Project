\documentclass[10pt,a4paper]{article}
\usepackage[utf8]{inputenc}
\usepackage{amsmath}
\usepackage{amsfonts}
\usepackage{amssymb}
\usepackage[margin=2.5cm]{geometry}
\author{Stefan Klaus, stk4@aber.ac.uk}
\title{Major Project outline specification }
\begin{document}
\bibliographystyle{annotate}
\maketitle
\begin{flushleft}
\hrulefill \\
\textbf{Project Title:} Map building using swarm intelligence \\[3ex]
\textbf{Supervisor:} Myra Wilson, mxw@aber.ac.uk\\[3ex]
\textbf{Degree scheme title and code:} AI \& Robotics, GH76\\[3ex]
\textbf{Module code:} CS39440\\[3ex]
\textbf{Version:} 1.0\\[3ex]
\textbf{Status:} Release\\
\hrulefill
\newpage

\section{Project Description }
The goal of my project is to be able to use swarm intelligence to map a target area. Some of the key points of this task are:

\begin{itemize}
\item networking between nodes(robots)
\item finding a good deployment solution for the swarm
\item mapping the area (including obstacle avoidance )
\item moving the swarm in unison while covering the largest possible area with the sensors
\item finding a mapping solution for rooms/corridors 
\end{itemize}

Each of this tasks has to be completed in order of getting a reasonably good solution with might be used to map unknown areas,however it does not need to be limited to mapping. The underlying key features: networking between the robots, deployment and obstacle avoidance techniques etc. can be used in all kind of applications which require robot flocking. One example could be to use robots equipped with cameras to get an inside view of potential dangerous buildings to asses the area before sending in humans. In my project however the solution will be limited to range finder and communication sensors to simply map the target area. I believe this is a worthwhile project as robot applications can gain greatly when used correctly in flocks.\\[3ex]

The main aspect of my project is to find a good deployment and communication solution for the flock, to assure that no units loose communication with the other and to be able to move the flock in unison while covering the largest possible area. \\
As I said above I am going to use the flock to map the area but theoretically a user can configure the robots to hold any kind of sensor e.g cameras. Given this and my plan to assure maximum area coverage while not having the need of a previous map or location information like GPS a user can then use the flock and their sensors for what ever task which requires wide area coverage.  \\
In my project I am going to use a limited number of robots however the flocking solution should be scalable enough for a flock of any size.\\[3ex]

The  mapping part of my project is going to use simple obstacle recognition and avoidance techniques which in the same time is going to be mapped. The problem with this is that I have to find a solution for map sharing between the robots as well as how to improve map accuracy. Each unit can easily map an area using distance sensors however to able to compare the map fragments with each other a global reference point(or a number of reference points) will be needed. \\
As my project is going to be simulator only this should not be a problem however as my indention is to make it program as applicable for the real world as possible as solution needs to be found. In one of the research papers I am using a solution using a millimetre wave radar was used to keep track of the global reference points while acquiring new obstacles\cite{citeulike:8530320}. As I planed to make my solution suitable for E-puck robots I will have to find a suitable solution using simpler techniques, however I do not have a clear idea how to do this as of this moment.\\[3ex]

My project is as stated before simulator based, this is for a sake of simplicity and the limited timespan we have to complete the project.  I am going to use the Webots\texttrademark robotic simulator and programs which work in the simulator should also work problem less on the E-puck robots (assuming real world restrictions are fulfilled i.e. the robots have the same sensor equipment). \\
The programming language I am going to use is C. \\
As this is a very research heavy project the information in this section is the outline as to date however I might have to change some aspects of it once I acquire new information and find newer, better solutions to a problem.
\newpage

\section{Proposed Tasks}
The simulator I am going to use is the Webots\texttrademark simulator which I will use to simulate a flock of E-puck robots. The programming language I am going to use is C. \\
My project is very research heavy and will perform research throughout the entire project. So far I have looked at research papers concerning SLAM (Simultaneous Localization and Map building)\cite{citeulike:8530320}, flock intelligence\cite{citeulike:130355} and networking between robots\cite{citeulike:4509551}. Theses gave me a general idea for a solution, however I have to look more into similar papers. \\[3ex]

The robots need to be able to deploy effectively while still staying in communication range to each other, as well covering the largest possible area. In a scenario where a number of different rooms exist a solution must be found to keep up communication, assuming that communication through walls is not possible. \\ 
For this kind of scenario it would be essential to assign roles to the different robots e.g. scout and communicator roles. Where the communicator stays in range of other communicators while the scouts map the area inside their communication range. This would mean that a room can be mapped by the scouts while the communicators stays inside the doorway and keeps contact with the rest of the flock.\\
I am at this moment not sure how to define scout and communicator roles as I image that it would be more useful to have all robots scouting in the beginning of a session and change then up into smaller "groups" which stay in contact through the communicators once the flock spreads out into corridor/room areas. I am not yet sure how to handle this problem yet and I am doing research on this matter.\\[3ex]

For the mapping part of the area I am going to use distances sensors for the simulated robots. I will also use this range data for obstacle avoidance techniques. The main problem here is to be able to define global reference points which will allow me to piece together the different map fragments I get from the different robots. As I for realism sake  do not have GPS information I will have to use the data from the range finders. I am not sure how to deal with the global reference point problem as of yet.\\
One possible solution would be to build the map dynamically on a central robot which would mean that all map data need to be transferred to it the moment it is acquired, this could lead to networking problems on larger flocks and I am doing research on to tackle this problem.\\[3ex]

\section{Project Deliverables}
\subsection{Final Program}
A working program which holds solutions to the deployment, communication and mapping problems. \\
The solution should be scalable for a flock of any size and a demonstration will show that the flock is able to deploy effectively and map the area as accurately as possible. 

\subsection{Source Code and Testing}
The complete, commented source code as well as any testing data which have been acquired. \\
The testing data will be part of the final documentation while the code can be delivered separately.\\
This also includes any world and robot files I created and used for the Webots\texttrademark simulator.

\subsection{Final Report}
The full report detailing the process as well as any changes to the design. This will include the design of the system and my reasoning for this design, the output data and detailed, analysed testing data. The testing analyse will hold all information I gather, problems I found during the testing and how I solved them. \\
It will hold a reference section of all sources I used for my research and the report will hold information about what information I gathered from them.  This will including dates of access for web pages. \\

\newpage

\bibliography{stefkla}


\end{flushleft}

\end{document}