\chapter{Background \& Objectives}
\label{Background}

This section should discuss your preparation for the project, including background reading, your analysis of the problem and the process or method you have followed to help structure your work.  It is likely that you will reuse part of your outline project specification, but at this point in the project you should have more to talk about. 

\textbf{Note}: 

\begin{itemize}
   \item All of the sections and text in this example are for illustration purposes. The main Chapters are a good starting point, but the content and actual sections that you include are likely to be different.
   
   \item Look at the document on the Structure of the Final Report for additional guidance. 
   
\end {itemize}

\subsection{SLAM - Simultaneous Localization And Mapping}
The SLAM problem is a current research topic which is based on different localisation algorithms and using a range of different sensor to effectively map an target area. A lot of different approaches have been done and many research papers have been written, the one this project is based on is a paper about a SLAM solution designed for autonomous vehicles\cite{Dissanayake2001Solution}.\\
While the research area of this paper is based on a much larger scale, it does still give me an insight upon the SLAM problem.\\
E.g. the problem with localisation in an dynamic environment, the paper tackles this problem by using global reference points and a millimetre wave radar, however for my project I do use an static environment and simple laser range finders. So this paper is only used a reference to the localisation problem, especially the idea of using "global" reference points for the created map.\\
As the test environment and the sensors available for the E-puck sensors are limited this project will assume that the starting location of the robots is known.\\[3ex]

Another paper which was read about this problem used an approach much more similar to this project, by using different mobile robots which have no GPS access and simply use 2D laser range finders. However the approach described in this paper was based around the mapping of one "lead" robot and the traversing the same map again with a second robot using the map generated by the first for localisation purposes.\\
The second robot would then scan the target area again and refine the already generated map though using the (now stationary) first robot as an reference point. Since this project is using single a robot for mapping purposes and multi robot usage would only be added if enough time is available, this paper was not inherently useful, however gave some useful insight for information sharing between robots or sensor stations as well as localisation of robots using a global reference point.\\
Since this project aims at real time localisation and mapping communication with a "uplink" point would be essential to achieve this in a real world setting. By using a similar implementation to the one described in the paper it would be possible to rescan a mapped area if it is traversed again, and by that refine the mapping. This is requires however very good localisation techniques.

\subsection{Deployment}
The deployment strategy is an important part of this project as it defines how effective the robot will cover the target area which will define how long it will take to scan and map the whole area. 
One research paper which was read proposed an solution of a communication network where the comm nodes keep track of the robots positions and guide them in directions which have not been explored in the last time period\cite{Batalin2003Coverage}. The paper uses a solution which is based on small comm nodes deployed by the robot, to make the solution fitting for this project the developer would have to use multiple E-Pucks and define some of these as communication nodes which remain on a fast position and guide the "scout" E-puck based on area which have been least visited by the other robots.\\ 
However since multi-robot usage is planned as an addition if enough time is available this deployment strategy is not fitting for the major part of work.\\[3ex]

\section{Analysis}
The background reading clearly showed the main aspects which were needed.\\
These were an effective deployment pattern, localisation of the robot and sensing of obstacles to map. 

\subsection{Deployment pattern}
An effective deployment pattern is needed as it is important to traverse the whole of the environment in order to map all obstacles inside it.\\
Characteristics for a effective deployment pattern are to traverse the whole environment and doing so in as short a time frame as possible.

\subsection{Localisation}
Arguably the most important aspect of the project is to have a good localisation solution, since without the mapping will be ineffective. There are a number of different approaches which can be done for this, papers which were researched showed approaches such as using global reference points, GPS and compass data as well as approaches which do not require data from sensors like GPS or compass modules.\\
One of the aspects wanted for this project is to be able to locate the robot without the use of GPS or compass data and rather use pure odometry calculations to reach a solution of the localisation problem.

\subsection{Mapping}
To be able to create a map of the environment the right sensing method needs to be found, i.e. different sensors perform different in similar environments, also the range of sensors varies a lot. 
The choice of sensors is limited as the work is based on the E-Puck robot platform.
However the quality of the mapping is strongly dependent on the deployment pattern and the localisation solution.

\section{Process}
The life cycle model used for this project is "Feature driven development", as it seams more appropriate for this project as other models. 

