\chapter{Code samples}
\label{code_samples}

\section{Moving forward a given distance}
\label{moving_forward_code}
This function moved the robot a given distance with a given speed.\\
The robot will slow down to a minimum speed on a minimum difference to the wanted position.\\
This method will update the global odometry information.\\
Description can be found in chapter \ref{Implementation} at section \ref{moving_forward_description} at page \pageref{moving_forward_description}.
\begin{lstlisting}[caption = {Moving forward}]
#define LEFT_DIAMETER 0.0416//orig 0.0416
#define RIGHT_DIAMETER 0.0404//orig 0.0404
#define WHEEL_DIAMETER (LEFT_DIAMETER + RIGHT_DIAMETER)
#define WHEELBASE 0.058 
#define INCREMENT_STEP 1000 //how many steps the motor takes for a full wheel rotation
#define MIN_DIST 20.0f //minimum difference to the new heading
#define MIN_SPEED 10.0f //speed when slowing down

#ifndef M_PI
	#define M_PI 3.1415926535897932384626433832795L
#endif

/**
Function to move the robot forward a given distance at a given speed
*/
void move_forward(double dSpeed, double dDist, struct odometryTrackStruct * ot){ 
	double dStepCount = 0.0f;
	double dStopPosLeft = 0.0f;
	double dStopPosRight = 0.0f;
	double *point_dEncPos;	
	
	if((dDist > 0.0f) && (dSpeed > 0.0f)){
		//calculate the number of steps
		dStepCount = (INCREMENT_STEP/(M_PI * WHEEL_DIAMETER / 2)) * dDist;
		
		/*read the current encoder positions of both wheels and
		calculate the encoder positions when the robot has to
		stop at the given distance ... */
		point_dEncPos = get_encoder_positions();
		
		dStopPosLeft = point_dEncPos[0] + dStepCount;
		dStopPosRight = point_dEncPos[1] + dStepCount;
		
		//compute odometry data
		odometry_track_step(ot);
		
		
		//set speed
		set_motor_speed(dSpeed, dSpeed); 
		
		//step tolereance test
		while((point_dEncPos[0] < dStopPosLeft) && (point_dEncPos[1] < dStopPosRight)){
			//get odometry data
			odometry_track_step(ot);
			//get wheel encoders
			point_dEncPos = get_encoder_positions();
			
			//slow down the closer the robot come to the destination, reduces the error
			if(fabs(dStopPosLeft - point_dEncPos[0]) <= MIN_DIST){
				set_motor_speed(MIN_SPEED, MIN_SPEED);
			}
		}
	}
	stop_robot();
	
	//update odometry data
	odometry_track_step(ot);
	wb_robot_step(TIME_STEP);
}
\end{lstlisting}

\section{Turning a given Angle}
\label{turning_angle_code}
This function turns the robot a given amount of degrees with a given amount of speed.\\
The robot will slow down to a minimum speed on a minimum threshold to the wanted heading.\\
Description can be found in chapter \ref{Implementation} at section \ref{turn_angle_description} at page \pageref{turn_angle_description}. 

\begin{lstlisting}[caption = {Turning an angle}]
#define LEFT_DIAMETER 0.0416
#define RIGHT_DIAMETER 0.0404
#define WHEEL_DIAMETER (LEFT_DIAMETER + RIGHT_DIAMETER)
#define WHEELBASE 0.058 
#define INCREMENT_STEP 1000 //how many steps the motor takes for a full wheel rotation
#define MIN_DIST 20.0f //minimum difference to the new heading
#define MIN_SPEED 10.0f //speed when slowing down

/**
Function to turn the robot a given angle with a given speed
*/
void turn_angle(double dAngle, double dSpeed){
	double dFactor = 0.0f;
	double dStepCount = 0.0f;
	double dStopPosLeft = 0.0f;
	double dStopPosRight = 0.0f;
	double *point_dEncPos;
	
	if((dAngle != 0.0f) && (dSpeed > 0.0f)){
		//calculate turn factor
		dFactor = fabs(360.0f/dAngle);
		
		//calculate the number of step counts for the rotations
		dStepCount = (INCREMENT_STEP * WHEELBASE)/(dFactor * WHEEL_DIAMETER / 2);
		
		point_dEncPos = get_encoder_positions();
		
		//turn right
		if(dAngle > 0){
			//calculate the target encoder positions
			dStopPosLeft = point_dEncPos[0] + dStepCount;
			dStopPosRight = point_dEncPos[1] - dStepCount;
			
			set_motor_speed(dSpeed, -dSpeed);
			
			while((point_dEncPos[0] < dStopPosLeft) && (point_dEncPos[1] > dStopPosRight)){
				
				//get wheel encoders
				point_dEncPos = get_encoder_positions();
				
				//slow down the closer the robot come to the destination, reduces the error
				if(fabs(dStopPosLeft - point_dEncPos[0]) <= MIN_DIST){
					set_motor_speed(MIN_SPEED, -MIN_SPEED);
				}
			}	
		
	} else { // turn left ...
		dStopPosLeft = point_dEncPos[0] - dStepCount;
		dStopPosRight = point_dEncPos[1] + dStepCount;

		// turn left the robot ...
		set_motor_speed(-dSpeed, dSpeed);

		while((point_dEncPos[0] > dStopPosLeft) &&(point_dEncPos[1] < dStopPosRight)){

			point_dEncPos = get_encoder_positions();
			if( fabs(dStopPosLeft - point_dEncPos[0]) <= MIN_DIST ){
				set_motor_speed(-MIN_SPEED, MIN_SPEED); }
			}
		}
	}
	stop_robot();

	wb_robot_step(TIME_STEP);
}

\end{lstlisting}

\section{Odometry Struct}
\label{odometry_struct_code}
This section shows the global odometry struct, which is used and updated throughout the program. 
\begin{lstlisting}[caption = {Odometry struct}]
 struct odometryTrackStruct {
	struct {
		float wheel_distance;
		float wheel_conversion;
	} configuration;
	struct {
		int pos_left_prev;
		int pos_right_prev;
	} state;
	struct {
		float x;
		float y;
		float theta;
	} result;
};
\end{lstlisting}

\section{Initializing odometry struct}
\label{odometry_init_code}
These functions are used to initialize the odometry struct and set its attributes.
Further description of this can be found in chapter \ref{Implementation} at section \ref{odometry_init_description} at page \pageref{odometry_init_description}.
\begin{lstlisting}[caption = {Initializing odometry struct}]
#define LEFT_DIAMETER 0.0416
#define RIGHT_DIAMETER 0.0404
#define WHEEL_DIAMETER (LEFT_DIAMETER + RIGHT_DIAMETER)
#define WHEELBASE 0.058 
#define INCREMENTS 1000.0 //how many steps the motor takes for a full wheel rotation
#define SCALING_FACTOR 1 

#ifndef M_PI
	#define M_PI 3.1415926535897932384626433832795L
#endif

/**
Initializes the odometry algortihms
*/
int odometry_track_start(struct odometryTrackStruct * ot){
	double* point_dEncPos;
	point_dEncPos = get_encoder_positions();
	return(odometry_track_start_pos(ot, point_dEncPos));
}

/**
Start the odometry tracking
Updates the info in the odometryTrackStruct for the first time
*/
int odometry_track_start_pos(struct odometryTrackStruct * ot, double* dEncPos){
	ot->result.x = 0;
	ot->result.y = 0;
	ot->result.theta = 0;
	
	ot->state.pos_left_prev = dEncPos[0];
	ot->state.pos_right_prev = dEncPos[1];
	
	ot->configuration.wheel_distance = axis_wheel_ratio * SCALING_FACTOR * (WHEEL_DIAMETER / 2);
	ot->configuration.wheel_conversion= (WHEEL_DIAMETER / 2) * SCALING_FACTOR * M_PI / INCREMENTS;
	
	return 1;
}
\end{lstlisting}

\section{Odometry step}
\label{odometry_update_code}
These functions are used to update the the odometry struct.
Further explanations of these functions can be found in chapter \ref{Implementation} at section \ref{odometry_update_description} at page \pageref{odometry_update_description}.
\begin{lstlisting}[caption = {Updating odometry struct}]
#ifndef M_PI
	#define M_PI 3.1415926535897932384626433832795L
#endif

/**
Updates an odometry data,
fetches the encoder positions and initialize the 
odoemetry_track_step_pos function
*/
void odometry_track_step(struct odometryTrackStruct * ot){
	double* point_dEncPos;
	
	point_dEncPos = get_encoder_positions();
	odometry_track_step_pos(ot,	point_dEncPos);
}

/**
Updates all odometry data for the struct.
Calculates the X and Y positions and orientation. 
*/
void odometry_track_step_pos(struct odometryTrackStruct * ot, double* dEncPos){
	int delta_pos_left, delta_pos_right;
	float delta_left, delta_right, delta_theta, theta2;
	float delta_x, delta_y;
	
	//calculate the difference in position between the previous and current encoder positions. 
	delta_pos_left = dEncPos[0] - ot->state.pos_left_prev;
	delta_pos_right = dEncPos[1] - ot->state.pos_right_prev;
	
	//calculate the rotation based on the displacement of the stepper motors 
	delta_left = delta_pos_left * ot->configuration.wheel_conversion;
	delta_right = delta_pos_right * ot->configuration.wheel_conversion;
	delta_theta = (delta_right - delta_left) / ot->configuration.wheel_distance;
	
	// calculate the x and y displacement 
	theta2 = ot->result.theta + delta_theta * 0.5;
	delta_x = (delta_left + delta_right) * 0.5 *cosf(theta2); 
	delta_y = (delta_left + delta_right) * 0.5 * sinf(theta2);
	
	//update the x, y and theta of the struct
	ot->result.x += delta_x;
	ot->result.y += delta_y;
	ot->result.theta += delta_theta;

	if(ot->result.theta >=361){
		odometry_track_step(ot);
	}
	
	if(ot->result.theta > M_PI){
		ot->result.theta -= 2*M_PI;
	}
	if(ot->result.theta < -M_PI){
		ot->result.theta += 2*M_PI;
	}
	
	//save current encoder positions to the global buffer 
	ot->state.pos_left_prev = dEncPos[0];
	ot->state.pos_right_prev = dEncPos[1]; 
}
\end{lstlisting}

\section{Reference point struct}
\label{ref_struct_code}
This section holds the code for the reference struct.\\
A further description of this struct can be found in chapter 3 section \ref{ref_struct_description} on page \pageref{ref_struct_description}
\begin{lstlisting}[caption={Reference point struct}]
 struct referencePos {
 	struct {
 		float x;
 		float y;
 	} lower_left;
 	struct {
 		float x;
 		float y;
 	} lower_right;
 	struct {
 		float x;
 		float y;
 	} upper_left;
 	struct {
		float x;
 		float y;
 	} upper_right;
 };
\end{lstlisting}

\section{Reference point check}
\label{ref_check_code}
This is the function which checks if a reference point has been reached and if so either saves it or resets the robot variables.\\
A more detailed explanation of this code can be found in chapter 3 section \ref{ref_check_description} on page \pageref{ref_check_description}.
\begin{lstlisting}[caption={Function which checks if a reference point has been reached}]
/**
This function checks the current robot position with the previous recorded reference position.
If a relation is found, the current encoder position are updated with the saved encoder postions.
The parameters are a pointer to the struct odometrtTrackStruct and refrencePos 
and an int which defines which corner it is. 
The corner is defined in the run() function where this function will be called.
1 = lower_left
2 = lower_right
3 = upper_left
4 = upper_right
*/
void check_reference_points(struct odometryTrackStruct * ot, struct referencePos * ref){
	double dThreshold = 20.0f;
	int i;
	double *point_dEncPos = get_encoder_positions(); 
	double dCurPosX = ot->result.x;
	int *point_SensorData;
	int corner = 0;
	int obstacle[NUM_DIST_SENS]={0,0,0,0,0,0,0,0};
	int ps_offset[NUM_DIST_SENS] = {35,35,35,35,35,35,35,35};
	int direction = check_direction(ot->result.theta);
	bool ob_front, ob_left, ob_right;
	char message[] = "Resetting";

/*	get distance sensor data*/
	point_SensorData = get_sensor_data(NUM_DIST_SENS);
	for(i = 0;i < NUM_DIST_SENS;i++){
		obstacle[i] = point_SensorData[i] - ps_offset[i] > THRESHOLD_DIST;
	}	

	ob_front  = 
	obstacle[0] ||
	obstacle[7];

	ob_right = 
	obstacle[2];

	ob_left = 
	obstacle[5];

	/*find out what corner it is */
	//north 
	if(direction == 1){
		if(ob_front && ob_left){
			corner = 3;
		}else if(ob_front && ob_right){
			corner = 4;
		}
	}

	//east
	if(direction == 2){
		if(ob_front && ob_left){
			corner = 4;
		}else if(ob_front && ob_right){
			corner = 2;
		}
	}

	//south
	if(direction == 3){
		if(ob_front && ob_left){
			corner = 2;
		}else if(ob_front && ob_right){
			corner = 1;
		}
	}

	//west
	if(direction == 4){
		if(ob_front && ob_left){
			corner = 1;
		}else if(ob_front && ob_right){
			corner = 3;
		}
	}

	/*Check if the current corner is set to 0, if so force update*/
	if(corner == 1){
		if((ref->lower_left.x == 0.00) && (ref->lower_left.y == 0.00)){
			set_reference_point(ot, ref, corner);
			return;
		}
	}else if(corner == 2){
		if((ref->lower_right.x == 0.00) && (ref->lower_right.y == 0.00)){
			set_reference_point(ot, ref, corner);
			return;
		}
	}else if(corner == 3){
		if((ref->upper_left.x == 0.00) && (ref->upper_left.y == 0.00)){
			set_reference_point(ot, ref, corner);
			return;
		}
	}else if(corner == 4){
		if((ref->upper_right.x == 0.00) && (ref->upper_right.y == 0.00)){
			set_reference_point(ot, ref, corner);
			return;
		}
	}

	/*If the corner is set updated the odometery information of the robot */
	if(direction == 1){ //north 
		if(((ref->upper_left.x <= dCurPosX + dThreshold) || (ref->upper_left.x >= dCurPosX - dThreshold)) && corner == 3){
			ot->result.x = ref->upper_left.x;
			ot->result.y = ref->upper_left.y;
			ot->state.pos_left_prev = point_dEncPos[0];
			ot->state.pos_right_prev = point_dEncPos[1]; 
			printf("%s at corner: %d with cords: %f %f\n", message, corner, ref->upper_left.x, ref->upper_left.y);
		}else if(((ref->upper_right.x <= dCurPosX + dThreshold )|| (ref->upper_right.x >= dCurPosX - dThreshold)) && corner == 4){
			ot->result.x = ref->upper_right.x;
			ot->result.y = ref->upper_right.y;
			ot->state.pos_left_prev = point_dEncPos[0];
			ot->state.pos_right_prev = point_dEncPos[1]; 
			printf("%s at corner: %d with cords: %f %f\n", message, corner, ref->upper_right.x, ref->upper_right.y);
		}
	}else if(direction == 2){//east 
		if(((ref->upper_right.x <= dCurPosX + dThreshold) || (ref->upper_right.x >= dCurPosX - dThreshold)) && corner == 4){
			ot->result.x = ref->upper_right.x;
			ot->result.y = ref->upper_right.y;
			ot->state.pos_left_prev = point_dEncPos[0];
			ot->state.pos_right_prev = point_dEncPos[1]; 
			printf("%s at corner: %d with cords: %f %f\n", message, corner, ref->upper_right.x, ref->upper_right.y);
		}else if(((ref->lower_right.x <= dCurPosX + dThreshold) || (ref->lower_right.x >= dCurPosX - dThreshold)) && corner == 2){
			ot->result.x = ref->lower_right.x;
			ot->result.y = ref->lower_right.y;
			ot->state.pos_left_prev = point_dEncPos[0];
			ot->state.pos_right_prev = point_dEncPos[1]; 
			printf("%s at corner: %d with cords: %f %f\n", message, corner, ref->lower_right.x, ref->lower_right.y);
		}
	}else if(direction == 3){ //south
		if(((ref->lower_left.x <= dCurPosX + dThreshold) || (ref->lower_left.x >= dCurPosX - dThreshold)) && corner == 1){
			ot->result.x = ref->lower_left.x;
			ot->result.y = ref->lower_left.y;
			ot->state.pos_left_prev = point_dEncPos[0];
			ot->state.pos_right_prev = point_dEncPos[1]; 
			printf("%s at corner: %d with cords: %f %f\n", message, corner, ref->lower_left.x, ref->lower_left.y);
		}else if(((ref->lower_right.x <= dCurPosX + dThreshold) || (ref->lower_right.x >= dCurPosX - dThreshold)) && corner == 2){
			ot->result.x = ref->lower_right.x;
			ot->result.y = ref->lower_right.y;
			ot->state.pos_left_prev = point_dEncPos[0];
			ot->state.pos_right_prev = point_dEncPos[1];
			printf("%s at corner: %d with cords: %f %f\n", message, corner, ref->lower_right.x, ref->lower_right.y);
		}
	}else if(direction == 4){ //west
		if(((ref->upper_left.x <= dCurPosX + dThreshold) || (ref->upper_left.x >= dCurPosX - dThreshold)) && corner == 3){
			ot->result.x = ref->upper_left.x;
			ot->result.y = ref->upper_left.y;
			ot->state.pos_left_prev = point_dEncPos[0];
			ot->state.pos_right_prev = point_dEncPos[1]; 
			printf("%s at corner: %d with cords: %f %f\n", message, corner, ref->upper_left.x, ref->upper_left.y);
		}else if(((ref->lower_right.x <= dCurPosX + dThreshold) || (ref->lower_right.x >= dCurPosX - dThreshold)) && corner == 1){
			ot->result.x = ref->lower_left.x;
			ot->result.y = ref->lower_left.y;
			ot->state.pos_left_prev = point_dEncPos[0];
			ot->state.pos_right_prev = point_dEncPos[1]; 
			printf("%s at corner: %d with cords: %f %f\n", message, corner, ref->lower_left.x, ref->lower_left.y);
			printf("Ot is: %f and %f", ot->result.x, ot->result.y);
		}
	}
}
\end{lstlisting}

\section{Saving a reference point}
\label{ref_save_code}
This section holds the code to save a new reference point to the reference struct.
An explanation of this code can be found in chapter 3 section \ref{ref_save_description} on page \pageref{ref_save_description}.

\begin{lstlisting}[caption={Function to save a new reference point to the reference struct}]
/**
Sets the reference point in the referencePos struct.
The parameters are a pointer to the referencePos struct and an int.
The int specifies which corner will be set.
1 = lower_left
2 = lower_right
3 = upper_left
4 = upper_right
*/
void set_reference_point(struct odometryTrackStruct * ot, struct referencePos * ref, int corner){
	char text[] = "Saving";
	printf("%s at corner: %d with cords: %f and %f\n", text, corner, ot->result.x, ot->result.y);
	if(corner == 1){
		ref->lower_left.x += ot->result.x;
		ref->lower_left.y += ot->result.y;
	}else if(corner == 2){
		ref->lower_right.x = ot->result.x;
		ref->lower_right.y = ot->result.y;
	}else if(corner == 3){
		ref->upper_left.x = ot->result.x;
		ref->upper_left.y = ot->result.y;
	}else if(corner == 4){
		ref->upper_right.x = ot->result.x;
		ref->upper_right.y = ot->result.y;
	}
}
\end{lstlisting}